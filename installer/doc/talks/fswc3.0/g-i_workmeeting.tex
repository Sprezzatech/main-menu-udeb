\documentclass{beamer}


\mode<presentation>
{
  \usetheme{Warsaw}
  % or ...

  \setbeamercovered{transparent}
  % or whatever (possibly just delete it)
}


\usepackage[english]{babel}
% or whatever

\usepackage{pdfpages}

\usepackage{ucs}
\usepackage[utf8x]{inputenc}
% or whatever

\usepackage{times}
\usepackage[T1]{fontenc}

\setbeamertemplate{navigation symbols}{}

\title[Extremadura worksessions - Graphical Installer] % (optional, use only with long paper titles)
{Graphical Installer worksession\\Casar de C\'{a}ceres\\January 2006}

\author[Frans Pop] % (optional, use only with lots of authors)
{Frans ~Pop \\ (Release Manager for Debian Installer) \\ \texttt{fjp@debian.org}}
% - Give the names in the same order as the appear in the paper.
% - Use the \inst{?} command only if the authors have different
%   affiliation.

% - Use the \inst command only if there are several affiliations.
% - Keep it simple, no one is interested in your street address.

\date[FSWC 3.0, 2007] % (optional, should be abbreviation of conference name)
{Free Software World Conference 3.0,\\Badajoz}
% - Either use conference name or its abbreviation.
% - Not really informative to the audience, more for people (including
% yourself) who are reading the slides online




\pgfdeclareimage[height=2cm]{debian-logo}{debian-swirl}
\logo{\pgfuseimage{debian-logo}}



% Delete this, if you do not want the table of contents to pop up at
% the beginning of each subsection:
\AtBeginSection[]
{
  \begin{frame}<beamer>
    \frametitle{Outline}
    \tableofcontents[currentsection]
  \end{frame}
}


% Delete this, if you do not want the table of contents to pop up at
% the beginning of each subsection:
%\AtBeginSubsection[]
%{
%  \begin{frame}<beamer>
%    \frametitle{Outline}
%    \tableofcontents[currentsection,currentsubsection]
%  \end{frame}
%}


% If you wish to uncover everything in a step-wise fashion, uncomment
% the following command: 
%\beamerdefaultoverlayspecification{<+->}


\begin{document}

\begin{frame}
  \titlepage
\end{frame}

\begin{frame}
  \tableofcontents
\end{frame}

%%%%%%%%%%%%%%%%%

\section{Introduction}

\begin{frame}
  \frametitle{Why a graphical installer?}
	\begin{itemize}
	\item
		Many users uncomfortable with textual interface
	\item
		Support for additional languages
	\item
		In principle not that hard to implement
	\end{itemize}
\end{frame}

\begin{frame}[plain]
  \frametitle{Modular structure}
  \pgfdeclareimage[width=11.5cm]{frontends}{frontends}
  \pgfuseimage{frontends}
\end{frame}

\begin{frame}[plain]
  \frametitle{The text frontend}
  \pgfdeclareimage[width=11cm]{text}{fe_text}
  \pgfuseimage{text}
\end{frame}

\begin{frame}[plain]
  \frametitle{The textual (newt) frontend}
  \pgfdeclareimage[width=11cm]{newt}{fe_newt}
  \pgfuseimage{newt}
\end{frame}

\begin{frame}[plain]
  \frametitle{The graphical (gtk) frontend}
  \pgfdeclareimage[width=11cm]{gtk}{fe_gtk}
  \pgfuseimage{gtk}
\end{frame}

\section{Status before the meeting}

\begin{frame}
  \frametitle{}
	\begin{itemize}
	\item
		Graphical frontend originally started December 2000
	\item
		Project abandoned; code unmaintained
	\item
		Picked up by Attilio Fiandrotti March 2005 (thesis)
	\item
		Initially mostly solo effort
	\item
		October 2005: mature enough to make images available
	\item
		Still heavy use of unofficial libraries (backports)
	\item
		Gradually more involvement and enthusiasm
	\end{itemize}
\end{frame}

\section{Worksession in Casar de C\'{a}ceres}

\begin{frame}
  \frametitle{Why a worksession?}
	\begin{itemize}
	\item
		Timing was very good
	\item
		Attilio was getting stuck on some technical issues
	\item
		Quality was getting good enough for release with Etch
	\item
		Challenges
		\begin{itemize}
		\item
			integration of DirectFB/GTK support in official libraries
		\item
			improvement of font management
		\item
			reduction of memory usage
		\item
			integration in official build system
		\end{itemize}
	\end{itemize}
\end{frame}

\begin{frame}
  \frametitle{Who were there?}
	\begin{itemize}
	\item
		Attilio Fiandrotti - Developer of the GTK frontend
	\item
		Frans Pop - DD, D-I Release Manager, build system
	\item
		Davide Viti - Fonts and coordination
	\item
		Sven Luther - DD, PowerPC porter
	\item
		Eddy Petrişor - Fonts, PowerPC tester
	\end{itemize}
	\begin{itemize}
	\item
		Denis Oliver Kropp - Upstream DirectFB developer
	\item
		Mike Emmel - Upstream DirectFB/GTK developer
	\end{itemize}
\end{frame}

\begin{frame}
  \frametitle{What was achieved?}
	\begin{itemize}
	\item
		Lot of discussion about goals, issues, ToDo
	\item
		Introduction into structure of D-I and build system
	\item
		GTK/DirectFB code merged into upstream GTK repository
	\item
		Investigation of some technical issues
	\item
		Improvement of build system (CD-ROM support)
	\item
		Some work on font and games
	\item
		Most importantly: team building, socializing
	\end{itemize}
\end{frame}

\begin{frame}
  \frametitle{Was it worth it?}
	\begin{itemize}
	\item
		Small, focused meeting
	\item
		Most important players were present
	\item
		Commitment from upstream developers
	\item
		Would we have been where we are now without the meeting?
	\end{itemize}
\end{frame}

\section{Conclusion}

\begin{frame}
  \frametitle{Well...}
	\begin{center}
		\huge{Thank you!}
	\end{center}
\end{frame}

\end{document}
