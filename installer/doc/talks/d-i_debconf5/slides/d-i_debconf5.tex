\documentclass{beamer}


\mode<presentation>
{
  \usetheme{Warsaw}
  % or ...

  \setbeamercovered{transparent}
  % or whatever (possibly just delete it)
}


\usepackage[english]{babel}
% or whatever

\usepackage{pdfpages}

\usepackage{ucs}
\usepackage[utf8x]{inputenc}
% or whatever

\usepackage{times}
\usepackage[T1]{fontenc}


\title[Some bits about the Debian Installer] % (optional, use only with long paper titles)
{Some bits about the Debian Installer}

\author[joeyh, h0lger, bubulle, fjp] % (optional, use only with lots of authors)
{Joey ~Hess, Holger ~Levsen, Christian ~Perrier, Frans ~Pop}
% - Give the names in the same order as the appear in the paper.
% - Use the \inst{?} command only if the authors have different
%   affiliation.

% - Use the \inst command only if there are several affiliations.
% - Keep it simple, no one is interested in your street address.

\date[Debconf 5] % (optional, should be abbreviation of conference name)
{5th Debian Conference, Helsinki,\\ Finland}
% - Either use conference name or its abbreviation.
% - Not really informative to the audience, more for people (including
% yourself) who are reading the slides online




\pgfdeclareimage[height=2cm]{debian-logo}{debian-swirl}
\logo{\pgfuseimage{debian-logo}}



% Delete this, if you do not want the table of contents to pop up at
% the beginning of each subsection:
\AtBeginSubsection[]
{
  \begin{frame}<beamer>
    \frametitle{Outline}
    \tableofcontents[currentsection,currentsubsection]
  \end{frame}
}


% If you wish to uncover everything in a step-wise fashion, uncomment
% the following command: 

\beamerdefaultoverlayspecification{<+->}


\begin{document}

\begin{frame}
  \titlepage
\end{frame}

%%%%%%%%%%%%%%%%%

\section{Debian Installer: past and present}

\begin{frame}
  \frametitle{}
\end{frame}

%%%%%%%%%%%%%%%%%

\section{Debian Installer internationalization and localization}

\subsection{Technical aspects}

\begin{frame}
  \frametitle{Design choices}
	\begin{itemize}
	\item
		English sucks
	\item
		All displayed texts must be translated
	\item
		Use of debconf
	\item
		Use of gettext (po-debconf)
	\item
		All translatable material to debconf templates
	\end{itemize}
\end{frame}

\begin{frame}
  \frametitle{Translatable material}
	\begin{itemize}
	\item
		Non sucking English
	\item
		63 packages 
	\item
		1260 strings
	\item
		40+1 languages
	\end{itemize}
\end{frame}

\begin{frame}
  \frametitle{The master file concept}
	\begin{itemize}
	\item
		One unique file to translate
	\item
		All PO files merged together
	\item
		Synchronisation black magic
	\end{itemize}
\end{frame}

\begin{frame}
  \frametitle{Maintainers changes propagation cycle}
	\begin{itemize}
	\item
		(maintainer) changes <package>/debian/<package>.templates
	\item
		(script) updates <package>/debian/po/template.pot
	\item
		(script) merges all template.pot files to po/template.pot
	\item
		(script) syncs po/*.po with the new templates.pot
	\item
		(script) updates status pages
	\item
		(Klingon translator) rants, down to 99\%, in tlh.po
	\end{itemize}
\end{frame}

\begin{frame}
  \frametitle{Translators changes propagation cycle}
	\begin{itemize}
	\item
		(Klingon translator) updates po/tlh.po to 100\%
	\item
		(script) propagates changes to all <package>/debian/po/tlh.po files	\item
		Sync po/*.po with the new templates.pot --> fuzzy/untranslated
	\item
		Update <package>/debian/po/*.po from the master files

	\item
		(script) updates status pages. tlh back to 100\%
	\item
		(Klingon translator) opens beer
	\end{itemize}
\end{frame}

\begin{frame}
  \frametitle{Levels of translation: now}
	\begin{itemize}
	\item
		Level 1: all ``core'' D-I packages (stage 1)
	\item
		Level 2: packages interacting with users during a default base system install
		\begin{itemize}
			\item
				base-config, tasksel
			\item
				exim4, shadow
			\item
				iso-codes, console-data
		\end{itemize}
	\item
		Level 3: packages interacting with users during any base system install
		\begin{itemize}
			\item
				pppconfig, popcon, localization-config, aptitude...
		\end{itemize}
		\item
			Level 4: packages displaying information to users during any base system install
		\begin{itemize}
			\item
				dpkg, apt, shadow...
		\end{itemize}
	\end{itemize}
\end{frame}

\begin{frame}
  \frametitle{Levels of translation: soon?}
	\begin{itemize}
	\item
		Level 5: packages interacting with users in a default desktop install
	\item
		Level 6: all other Debian native packages/debconf translation of stadard+base packages
	\item
		Level 7: English kicked out of Debian
	\end{itemize}
\end{frame}

\begin{frame}
  \frametitle{Status pages}
	\begin{itemize}
	\item
		Easy picture of overall l10n status 
	\item
		Points to packages CVS/SVN/TLA/BAZ/FOO/BAR/... repositories
	\item
		Give easy access to up-to-date POT and PO files
	\end{itemize}

\url{http://people.debian.org/~seppy/d-i/translation-status.html}

\end{frame}

\begin{frame}
  \frametitle{Quality assurance}
	\begin{itemize}
	\item
		Peer review : the role of translation teams
	\item
		Otaumated spelchaicking and comonne errors tracking
		\begin{itemize}
			\item
				\url{http://d-i.alioth.debian.org/spellcheck/level1-post-sarge/index.html}
		\end{itemize}
	\item
		Test localized installations (hint: D-I works well with QEmu!)
	\end{itemize}

\url{http://people.debian.org/~seppy/d-i/translation-status.html}

\end{frame}

%%%%%%%%%%%%%%%%%

\section{Installation Guide}

\begin{frame}
  \frametitle{}
\end{frame}

%%%%%%%%%%%%%%%%%

\section{The future of Debian Installer: how to help}

\begin{frame}
  \frametitle{}
\end{frame}



\end{document}
 	
